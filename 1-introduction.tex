\section{Einführung}

\begin{takeaway}
    \item NP-schwer vs. NP-vollständig
    \item Entscheidungsproblem
    \item Schwellwertsprache
    \item Diverse Notationen, Definitionen und Beispiele
\end{takeaway}

\paragraph{Polynomzeit-Reduzierbarkeit}
Ein Entscheidungsproblem $\Pi_1$ ist ``polynomzeit-reduzierbar'' auf ein anderes
Entscheidungsproblem $\Pi_2$:
\begin{align*}
& \Longleftrightarrow
\exists \text{ Algo } \A \text{ s.t. } \Time_\A \in \text{poly}
\wedge \Pi_1(x) = \Pi_2(\A(x))
\\
& \Longleftrightarrow
\Pi_2 \text{ mindestens so schwer wie } \Pi_1
\\
& \Longleftrightarrow
\Pi_1 \text{ höchstens so schwer wie } \Pi_2
\\
& \Longleftrightarrow
\Pi_1 \preceq_P \Pi_2
\end{align*}

\paragraph{Klasse NP}
Nichtdeterministisch-Polynom-Zeit.
Klasse der Probleme, die sich in Polynomzeit verifizieren lassen, und von einer nichtdeterministischen
Turing-Maschine in Polynomzeit lösen lassen (z.B. durch Lösung raten und verifizieren).

\paragraph{NP-schwer (NP-hard)}
Ein Problem $\Pi$ das ``mindestens so schwer'' ist wie alle Probleme in NP.
D.h. alle Probleme in NP lassen sich auf $\Pi$ reduzieren:
$$ \forall \Pi' \in NP : \Pi' \preceq_P \Pi $$
$\Pi$ muss nicht notwendigerweise in NP liegen (d.h. kann schwerer sein)!
Beispiel: das Halteproblem (nicht entscheidbar, daher $\notin NP$).

\paragraph{NP-vollständig (NP-complete)}
Ein Problem $\Pi$, das in NP liegt \underline{und} NP-schwer ist.
``Repräsentativ'' für die Menge NP, da sich alle Probleme aus NP darauf reduzieren lassen. \\
Beispiel: Satisfiability-Problem SAT (Satz von Cook).

\begin{figure}[h]
    \centering
    \includegraphics[scale=0.4]{images/np-hard-complete.png}
    \caption{Mengendiagramm der Beziehungen (Quelle: \href{https://commons.wikimedia.org/w/index.php?curid=3532181}{Wikipedia})}
    \label{fig:np-hard-complete}
\end{figure}

\paragraph{``Schwere Probleme''}
meisst NP-schwere Probleme, aber generell alle Probleme die sich nicht in Polynomzeit lösen lassen.
\textbf{Sinnvollerweise gehen wir im Folgenden davon aus dass $P \neq NP$.}

Allgemein sind Instanzen unseres Problems \underline{nicht} deterministisch in Polynomzeit lösbar\footnote{Es kann sein, dass einige instannzen in polynomzeit lösbar sind, siehe z.B. \emph{Win-Win-Algorithmen}.}.
Mögliche Ansätze:
\begin{enumerate}[label=\alph*)]
    \item nicht exakt sondern approximativ lösen (Approximationsalgorithmen)
    \item nicht deterministisch sondern nichtdeterministisch lösen (Randomisierte Algorithmen)
    \item nicht polynomiell sondern moderat exponentiell lösen%
    \footnote{D.h. die Basis der Exponentation ist klein, z.B. $1.4^n$ statt $2^n$.}
    \item nicht alle sondern Instanzen mit einer bestimmten Struktur lösen (Parametrisierte Algorithmen)
    \item anderweitig zusätzliche Informationen über die Eingabe nutzen (Reoptimierung, Win-Win-Strategy)
    \item Heuristiken\footnote{Nachteil: Im Gegensatz zu den anderen Ansätzen ist hier die Qualität (Laufzeit, ...) nicht beweisbar.}
\end{enumerate}


\subsection{Definitionen}

\paragraph{Entscheidungsproblem}
$P = (L, U, \Sigma)$ wobei
\begin{itemize}
    \item $\Sigma$ ein Alphabet
    \item $U \subseteq \Sigma^*$ die Menge der zulässigen Eingaben (als Wörter über dem Alphabet, als \emph{Sprache})
    \item $L \subseteq U$ die Menge der akzeptierten Eingaben (\emph{JA-Instanzen})
\end{itemize}
Ein Algorithmus $\A$ \emph{löst} $P$ falls gilt:
$$ \forall u \in U : A(x) =
\begin{cases}
1 \text{ oder JA}, \text{ if } x \in L \\
0 \text{ oder NEIN}, \text{ if } x \in U-L \\
\end{cases}
$$

\paragraph{Vertex Cover Problem VC}
``Der Hefepilz der parametrisierten Algorithmiker -- ein Modellproblem.''
\\
Eingabe $U$: ungerichteter Graph $G = (V, E)$ und $k \leq |V|, k \in \N$. \\
Ausgabe $L$: JA falls $\exists C \subseteq V$ s.t. $|C| \leq k$ mit $\forall \{u, v \} \in E: u \in C \vee v \in V$.\\
In Worten: Entscheide ob es ein Vertex Cover der Grösse $\leq k$ für einen gegebenen Graphen $G$ gibt.

\paragraph{Satisfiability-Problem SAT} \mbox{} \\
Eingabe: CNF-Formel $\Phi = C_1 \wedge \dots \wedge C_m$ mit Klauseln $C_i$ über Variablen $x_1, \dots, x_n$. \\
Ausgabe: eine Variablen-Belegung die $\Phi$ erfüllt.

Bei $l$-SAT enthält jede Klausel maximal $l$ Literale.

\paragraph{Optimierungsproblem}
$U = (L, M, cost, goal)$ wobei
\begin{itemize}
    \item $L$ die Sprache der zulässigen Eingaben\footnote{Oben noch $U$!}
    \item $\M: L \mapsto \Sigma^*$ so dass $M(x)$ die Sprache der akzeptierten Lösungen für Eingabe $x$
    \item $cost$: $\forall x \in L \; \forall y \in M(x) : cost(y, x) = $ Kosten der Lösung $y$ für Eingabe $x$
    \item $goal \in \{ \min, \max \}$ das Optimierungsziel
    \item $Opt_U(x) = goal \{ cost(y, x) | y \in M(x) \}$ die Kosten einer optimalen Lösung für Eingabe $x$
\end{itemize}

\paragraph{Minimum Vertex Cover Problem MIN-VC}
Wie VC, mit $cost(C, G) = |C| = $ Grösse des Vertex Covers und $goal = \min$.

\paragraph{MAX-SAT}
Wie SAT, mit $cost = $ Anzahl belegte Variablen und $goal = \max$.

\paragraph{Laufzeit}
eines Algorithmus' $\A$ auf Eingabe $x$ ist $\Time_\A (x)$
wobei $\Time_\A: \N \mapsto \N$.
Die Laufzeit von $\A$ in Abhängigkeit von der Grösse $n$ der Eingabe ist:
$\Time_\A (n) = \max \{ \Time_\A (x) \; | \; |x| = n, x \in L \}$.
Die Laufzeit wird in  $\bigO$-Notation angegeben.

\paragraph{Schwellwertsprache (threshold language)} 
definiert für ein Optimierungsproblem $U$ als die Sprache die 
für jede legale Eingabe von $U$ alle Tupel enthällt mit der Eingabe 
selber und einer Zahl (in Binär) die grösser gleich der Kosten der optimalen Lösung ist:

\[ Lang_U = \{ (x, a) \in L \times \binarystring \; | \; Opt_U(x) \leq \text{number}(a) \} \]
\[ Lang_U = \{ (x, a) \in L \times \binarystring \; | \; Opt_U(x) \geq \text{number}(a) \} \]

wo $Number(a)$ die Zahl mit der Binärdarstellung $a$ (der 
\emph{Schwellwert}) ist und jeweils $goal = \min/\max$. 

\textbf{Beispiel:} $Lang_{MIN-VC}$ enhällt für einen Eingabe $x$ die 
optimal mit einem VC der grösse $k$ alle Tupel $(x, k')$ mit $k' \geq k$. 
Somit gilt z.B. auch, dass $Lang_{MIN-VC} = VC$, weil wir die 
Entscheidung des $VC$ Problems herausfinden können, indem wir 
nach dem entsprechenden Tupel in $Lang_{MIN-VC}$ suchen. Anders 
herum können wir heraus finden ob ein Tupel $(x, a) \in Lang_{MIN-VC}$ 
indem wir $VC$ mit $k = a$ aufrufen.

\emph{Achtung}, diese Relation gilt nicht für alle Probleme, z.B. 
$Lang_{MAX-SAT} \neq SAT$ (da SAT leichter sein kann)!

$U$ heisst ``NP-schwer'' falls $Lang_U$ NP-schwer ist (warum?).%
\footnote{Wir erinnern uns, dass NP-Schwere für Entscheidungsprobleme 
definiert ist. Über die Schwellwertsprache erweitern wir dies für 
Optimierungsprobleme, indem wir aus dem Optimierungsproblem ein 
Entscheidungsproblem konstruieren.}

\paragraph{Abschätzen von Anzahl Teilmengen}
Frage: wie viele Teilmengen $T \subseteq \{1, \dots, n\}$ mit $|T| \leq k$ gibt es?
Grobe Abschätzung:
$$ \sum_{0 \leq i \leq k} \binom{n}{i}
\overset{\star}{\leq} \sum_{0 \leq i \leq k} n^i
\overset{\star\star}{\leq} \frac{n^{k+1} - 1}{n-1} \in \bigO(n^k) $$
Wobei $\star$ für kleine $k$ okay, aber grosse $k$ ungenau ist, und $\star\star$ 
eine geometrische Summe.

Da die Teilmengen in lexikographischer Ordnung durchlaufen werden können, 
lassen sie sich in amortisiert
$\bigO(n^k)$ enumerieren.
